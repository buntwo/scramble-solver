\newsavebox{\gameboard}
\savebox{\gameboard}{
  \begin{tikzpicture}[thick,scale=1]
    \setcounter{boardindex}{0}
    % Here we first draw a grid to help us debug.
    % The grid is drawn here:
    \ifthenelse{\isundefined{\debugme}}
    {}
    {\draw[step=1cm,gray,very thin] (-10,-10) grid (10,10);}

    % Cartesian coordinates are added here:
    \ifthenelse{\isundefined{\debugme}}
    {}
    {
    \foreach \x in {-9,...,9}
    \foreach \y in {-9,...,9}
    {
      \draw (\x, \y) node {{\tiny (\x, \y)}};
    }
    }
    
    % Calculate the dimensions of each square. 
    % We need to draw a 4x4 tableau, such that it is
    % horizontally and vertically centered at
    % the origin, and such that the each cell in
    % tableau is as large as possible.
    %
    % We require each cell to have an integral
    % dimension. Thus, the dimension is equal to
    % celldimension = 
    % min( floor( (\botx - \topx)/4 ), floor( (\topy - \boty)/4 ))
    
    % Uncomment below to regenerate ....
    % +++++++++++++++++++++++++++++++++++
    %\pgfmathparse{min( floor( (\botx - \topx)/4 ), floor( (\topy -
    %  \boty)/4 ))};
    %\let\celldimension\pgfmathresult
    % +++++++++++++++++++++++++++++++++++
    

    % Now we draw the tableau, starting at
    % top left = (-2 * \celldimension, 2 * \celldimension)
    % and ending at
    % bottom right = (2 * \celldimension, -2 * \celldimension)

    % We loop unroll instead to make this faster.
    % +++++++++++++++++++++++++++++++++++
    \foreach \x in {-2,-1, 0, 1}
    \foreach \y in {2,1, 0, -1}
    {
      \draw (\x * \celldimension, \y * \celldimension)
      rectangle +(\celldimension, -\celldimension);
    }
    % +++++++++++++++++++++++++++++++++++

    % Now we fill in the board text. Each character is to be centered
    % in each box, using the formulas:
    % 
    % mid_x = (right_x + left_x)/2
    % mid_y = (above_y + below_y)/2
    %
    % The board text is taken from the global array \BOARD. The
    % variable boardindex will index which character to put in the
    % current cell.
    %

    {
    \letterfont

    % Go through the tableau and draw the board
    \foreach \y in {2,1, 0, -1}
    \foreach \x in {-2,-1, 0, 1}
    {
      \pgfmathparse{(\x + .5) * \celldimension};
      \let\midx\pgfmathresult

      \pgfmathparse{(\y - .5) * \celldimension};
      \let\midy\pgfmathresult
      
      \draw (\midx, \midy) node 
      {\pgfmathparse{\BOARD[\arabic{boardindex}]}\pgfmathresult};
      
      \stepcounter{boardindex}
    }
    }
  \end{tikzpicture}
}

\newsavebox{\negone}
\newsavebox{\one}
\newsavebox{\negthree}
\newsavebox{\three}
\newsavebox{\negfour}
\newsavebox{\four}
\newsavebox{\negfive}
\newsavebox{\five}


\savebox{\negone}{
  \begin{tikzpicture}
    \draw[-\arrowheadstyle, 
    line width=\arrowlinewidth,
    color=\arrowcolor,
    opacity=\startingopacity]
    (0,0) -- +(-\celldimension, 0) node[thin, right] {};
  \end{tikzpicture}
}

\savebox{\one}{
  \begin{tikzpicture}
    \draw[-\arrowheadstyle, 
    line width=\arrowlinewidth,
    color=\arrowcolor,
    opacity=\startingopacity]
    (0,0) -- +(\celldimension, 0) node[thin, right] {};
  \end{tikzpicture}
}

\savebox{\negthree}{
  \begin{tikzpicture}
    \draw[-\arrowheadstyle, 
    line width=\arrowlinewidth,
    color=\arrowcolor,
    opacity=\startingopacity]
    (0,0) -- +(\celldimension, \celldimension) node[thin, right] {};
  \end{tikzpicture}
}

\savebox{\three}{
  \begin{tikzpicture}
    \draw[-\arrowheadstyle, 
    line width=\arrowlinewidth,
    color=\arrowcolor,
    opacity=\startingopacity]
    (0,0) -- +(-\celldimension, -\celldimension) node[thin, right] {};
  \end{tikzpicture}
}

\savebox{\negfour}{
  \begin{tikzpicture}
    \draw[-\arrowheadstyle, 
    line width=\arrowlinewidth,
    color=\arrowcolor,
    opacity=\startingopacity]
    (0,0) -- +(0, \celldimension) node[thin, right] {};
  \end{tikzpicture}
}

\savebox{\four}{
  \begin{tikzpicture}
    \draw[-\arrowheadstyle, 
    line width=\arrowlinewidth,
    color=\arrowcolor,
    opacity=\startingopacity]
    (0,0) -- +(0, -\celldimension) node[thin, right] {};
  \end{tikzpicture}
}

\savebox{\negfive}{
  \begin{tikzpicture}
    \draw[-\arrowheadstyle, 
    line width=\arrowlinewidth,
    color=\arrowcolor,
    opacity=\startingopacity]
    (0,0) -- +(-\celldimension, \celldimension) node[thin, right] {};
  \end{tikzpicture}
}

\savebox{\five}{
  \begin{tikzpicture}
    \draw[-\arrowheadstyle, 
    line width=\arrowlinewidth,
    color=\arrowcolor,
    opacity=\startingopacity]
    (0,0) -- +(\celldimension, -\celldimension) node[thin, right] {};
  \end{tikzpicture}
}

\newsavebox{\initcell}
\newsavebox{\finalcell}

\savebox{\initcell}{
  \begin{tikzpicture}
    \filldraw[initialcellstyle] 
    (0,0)
    rectangle
    +(\celldimension, -\celldimension);
  \end{tikzpicture}
}

\savebox{\finalcell}{
  \begin{tikzpicture}
    \filldraw[finalcellstyle] 
    (0,0)
    rectangle
    +(\celldimension, -\celldimension);
  \end{tikzpicture}
}